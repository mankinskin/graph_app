\chapter{Induction Algorithm}\label{chp:induction_algorithm}
%\addcontentsline{toc}{chapter}{#1}

%In this chapter we describe the algorithms for constructing the hierarchical n-gram grammar and applying it as a language model. First we describe the target structure and how it can be used to predict completions for a given sequence, then we describe an algorithm for inducing the structure from a single input string or a set of strings.

In this chapter we describe the fundamental algorithms required to implement the automatic induction procedure (described in~\ref{sec:induction}) and how to construct the context graph defined in the previous chapter.

\noindent
The induction procedure is divided into three main operations:

\begin{itemize}
\item \textbf{Search} Parses an input string to a parse tree in the context graph corresponding to the largest matching prefix. Produces a sub-graph annotated with positions to locate the exact range of the match. 
\item \textbf{Insert} Uses the parsed sub-graph from the search step to split and modify the context graph to guarantee a dedicated node for the matched string.
\item \textbf{Read} Repeatedly inserts nodes for largest known sub-strings found in an input string and consequently constructs a node representing the entire input string.
\end{itemize}

\subfile{alg_search}

\subfile{alg_insert}

\subfile{alg_induction}
