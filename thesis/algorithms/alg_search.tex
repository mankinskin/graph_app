
\section{Search}

The search algorithm finds the location of a given input string (called \emph{query}) in a given context graph structure. The entire query may not exist in the graph, therefore only the largest matching prefix is located. The algorithm constructs a \emph{trace graph} containing the nodes and edges that were traversed during search in the complete graph. The trace graph is an augmented sub-graph of the original graph which is rooted at the smallest vertex containing all the matching string with  paths to the locations of the beginning and the end of the matched string in the vertex.

\dfn{Trace Graph}{
    The trace graph $T_G$ of string $w \in \Sigma^*$ in context graph $G$ is a rooted sub-graph found in $G$, representing $w$. The edges of\ $T_G$ contain at least one path starting from the first node, passing through the root node and ending in the last node representing sub-strings of the query (mirroring traversal order of the search algorithm).\par
    The edges can be divided into \emph{upward} and \emph{downward} edges, where upward edges point to larger nodes and downwards edges point to smaller nodes.
    \begin{align*}
        \text{d}(a, b) =
        \begin{cases}
            \texttt{up}, &\text{ if } |a| < |b|\\
            \texttt{down}, &\text{ if } |a| > |b|\\
            \texttt{undefined}, &\text{ if } |a| = |b|\\
        \end{cases}
    \end{align*}
    \noindent
    Nodes of the same size can not be in a strict containment relation and thus no edges between exist in the context graph.

    Edges are annotated with a distance $\text{dist}(e)$ to the first token, relative to the nodes they connect.
    \begin{align*}
        \text{dist}((a, b)) =
        \begin{cases}
            \text{prefix length before }w_0 \text{ in } a, &\text{ if } (a, b) \text{ is upwards edge,}\\
            \text{distance from }w_0 \text{ to beginning of } b, &\text{ if } (a, b) \text{ is downwards edge}\\
        \end{cases}
    \end{align*}
    The distance values are later used to determine the relative positions a node needs to be split at to construct new nodes from partial overlaps with other nodes.
}
\noindent
To describe the search algorithm, let us assume there is an existing context graph structure that is correct according to our definition~\ref{chp:model}. From our definition, we can derive assumptions about the structure at every operation of the algorithm. For a base example of the graph structure we can give the graph containing a single sequence without repetitions.

\noindent 
The search query is given as a string of vertex indices $q \in {I_V}^*$, representing a string of either unigram tokens or larger n-gram nodes, meaning we are not limited to raw string search queries. The query is the pattern we are trying to find in the graph.

\noindent
The search algorithm maintains a queue of states with which it traverses all parse states in a breadth-first manner. The states contain information about the locations in both the $query$ and the graph structure $G$ to be compared. While traversing the graph, the algorithm populates the trace graph $T$ with new edges to each matched sub-string. The edges are annotated with relative positions which are later used in the insertion operation~\ref{sec:insert} to split any partially matching vertices for further processing.

\noindent
The algorithm can be understood as a parsing algorithm, matching the ``rules'' of graph structure on the query. This is a more intuitive description:
\begin{enumerate}
    \item Start at the first vertex index in the query $q_0$ and begin upwards traversal in $G$ in a breadth-first manner adding next nodes to a queue
    \item Stop at each parent where there exists a successor to the previous child edge, i.e.\ the parent describes a context of $q_0$ with consecutive tokens
    \item Compare the successive vertex $n$ with the following token in the query $q_n$.\ There are three cases for $n$ and $q_n$:
    \begin{itemize}
        \item equal: a matching transition has been found
        \item same length but different nodes: a mismatching transition has been found
        \item different lengths: continue search by comparing the prefixes of the larger vertex $l$ with the smaller vertex $s$. If $s$ is a prefix of $l$, there must be a path from $l$ to $s$.  
    \end{itemize}
    If both nodes have different lengths ($l$ and $s$ exist), the matching algorithm continues to match the prefixes of $l$ with $s$.
    Once we encounter a prefix $p$ of $l$ that is smaller than $s$, we can assume, based on the correct structure of $l$, that $s$ does not exist completely as a prefix of $l$ (because there is no path from $l$ to $s$ at the beginning). However, it is still possible that the prefix $p$ and $s$ both share a smaller prefix before their strings differ. To find these prefixes, $s$ becomes the new $l$ and $p$ becomes the new $s$ and the algorithm repeats until $s$ and $l$ are the same length, which is bound to happen as all vertices have a smallest prefix of length $1$, when a decision can be made.
    The result of this matching is therefore always a path from the parent node $n$ to a matching or mismatching prefix for both the query and the search cache.
\end{enumerate}

\noindent
This algorithm is repeated on larger and larger root nodes, until a root arrives at a mismatch with the query. During iteration, edges are added to the trace graph if they are guaranteed to be part of the matching sub-graph structure. Such is the case when a parent node has been found to match until its end and when a mismatch has been found. The collected paths and relative positions can then be translated to token locations in parallel partitions in each node, constructing all possible paths from the root node to the respective beginning and end tokens.

The resulting trace graph contains all the possible paths from the start vertex ($q_0$) to the root, aswell as from the root to the last vertex matching with the query. Each edge in the trace graph is an edge from the containment relation, annotated with a distance from the beginning of the query. These annotations are added during parsing and allow us to calculate the relative location of each border (start/end) inside each vertex.

\noindent
In the next step, this information is used to split each vertex into parts which are then recombined into new vertices.

%\begin{lstlisting}
%struct State {
%    key: int,
%    path: Path,
%    query_path: QueryPath,
%}
%\end{lstlisting}

%\begin{algorithm}
%    \caption{SEARCH}\label{alg:search}
%    \begin{flushleft}
%        \textbf{Input:} $[\texttt{VertexIndex}]\ query = q_0, \ldots q_n,\ \texttt{Grammar}\ G$\\
%        \textbf{Output:} $\texttt{TraceGraph}$
%    \end{flushleft}
%    \begin{algorithmic}
%        \State$trace \gets \texttt{TraceGraph}()$\Comment{Initialization}
%        \State$query\_path \gets \texttt{QueryPath}(query)$%
%
%        \State$queue \gets []$%
%        \State{$queue.\texttt{append}(\text{parent states for }q_0 \text{ with key }q_0.width)$}
%        \While{$queue$ has next $state$}
%            \State{$sk \gets state.key$}
%            \While{$state$ has successor $succ$}
%                \If{$query\_path$ has matching prefix in $succ$}
%                    \State{$sk \gets sk + s.width$}\Comment{found matching parent}
%                    \State{add path to $state.root$ to $trace$}
%                    \State{clear $queue$}
%                \ElsIf{$sk > k$}
%                    \State{add path from $state.root$ to last matching prefix to $trace$}
%                    \State$\texttt{return } trace$
%                \EndIf%
%            \EndWhile%
%            \State{append parent states of $p$ with key $sk$ to $queue$}
%        \State{$queue.\texttt{append}(\text{parent states for }p\text{ with key }sk)$}
%        \EndWhile%
%        \State$\texttt{return } trace$
%    \end{algorithmic}
%\end{algorithm}


%\begin{figure}
%\centering
%\begin{tikzpicture}[]
%    \graph [layered layout, nodes=draw, fresh nodes]{
%        {AB, AB, AB} -> AB -> {
%            AB -> {A, B},
%            BA -> {A, B}
%        };
%    };
%    \node [draw, fit= (AB) (AB') (AB'')] {};
%\end{tikzpicture}
%
%\caption{An example graph structure}
%\end{figure}
